
% Default to the notebook output style

    


% Inherit from the specified cell style.




    
\documentclass[11pt]{article}

    
    
    \usepackage[T1]{fontenc}
    % Nicer default font (+ math font) than Computer Modern for most use cases
    \usepackage{mathpazo}

    % Basic figure setup, for now with no caption control since it's done
    % automatically by Pandoc (which extracts ![](path) syntax from Markdown).
    \usepackage{graphicx}
    % We will generate all images so they have a width \maxwidth. This means
    % that they will get their normal width if they fit onto the page, but
    % are scaled down if they would overflow the margins.
    \makeatletter
    \def\maxwidth{\ifdim\Gin@nat@width>\linewidth\linewidth
    \else\Gin@nat@width\fi}
    \makeatother
    \let\Oldincludegraphics\includegraphics
    % Set max figure width to be 80% of text width, for now hardcoded.
    \renewcommand{\includegraphics}[1]{\Oldincludegraphics[width=.8\maxwidth]{#1}}
    % Ensure that by default, figures have no caption (until we provide a
    % proper Figure object with a Caption API and a way to capture that
    % in the conversion process - todo).
    \usepackage{caption}
    \DeclareCaptionLabelFormat{nolabel}{}
    \captionsetup{labelformat=nolabel}

    \usepackage{adjustbox} % Used to constrain images to a maximum size 
    \usepackage{xcolor} % Allow colors to be defined
    \usepackage{enumerate} % Needed for markdown enumerations to work
    \usepackage{geometry} % Used to adjust the document margins
    \usepackage{amsmath} % Equations
    \usepackage{amssymb} % Equations
    \usepackage{textcomp} % defines textquotesingle
    % Hack from http://tex.stackexchange.com/a/47451/13684:
    \AtBeginDocument{%
        \def\PYZsq{\textquotesingle}% Upright quotes in Pygmentized code
    }
    \usepackage{upquote} % Upright quotes for verbatim code
    \usepackage{eurosym} % defines \euro
    \usepackage[mathletters]{ucs} % Extended unicode (utf-8) support
    \usepackage[utf8x]{inputenc} % Allow utf-8 characters in the tex document
    \usepackage{fancyvrb} % verbatim replacement that allows latex
    \usepackage{grffile} % extends the file name processing of package graphics 
                         % to support a larger range 
    % The hyperref package gives us a pdf with properly built
    % internal navigation ('pdf bookmarks' for the table of contents,
    % internal cross-reference links, web links for URLs, etc.)
    \usepackage{hyperref}
    \usepackage{longtable} % longtable support required by pandoc >1.10
    \usepackage{booktabs}  % table support for pandoc > 1.12.2
    \usepackage[inline]{enumitem} % IRkernel/repr support (it uses the enumerate* environment)
    \usepackage[normalem]{ulem} % ulem is needed to support strikethroughs (\sout)
                                % normalem makes italics be italics, not underlines
    

    
    
    % Colors for the hyperref package
    \definecolor{urlcolor}{rgb}{0,.145,.698}
    \definecolor{linkcolor}{rgb}{.71,0.21,0.01}
    \definecolor{citecolor}{rgb}{.12,.54,.11}

    % ANSI colors
    \definecolor{ansi-black}{HTML}{3E424D}
    \definecolor{ansi-black-intense}{HTML}{282C36}
    \definecolor{ansi-red}{HTML}{E75C58}
    \definecolor{ansi-red-intense}{HTML}{B22B31}
    \definecolor{ansi-green}{HTML}{00A250}
    \definecolor{ansi-green-intense}{HTML}{007427}
    \definecolor{ansi-yellow}{HTML}{DDB62B}
    \definecolor{ansi-yellow-intense}{HTML}{B27D12}
    \definecolor{ansi-blue}{HTML}{208FFB}
    \definecolor{ansi-blue-intense}{HTML}{0065CA}
    \definecolor{ansi-magenta}{HTML}{D160C4}
    \definecolor{ansi-magenta-intense}{HTML}{A03196}
    \definecolor{ansi-cyan}{HTML}{60C6C8}
    \definecolor{ansi-cyan-intense}{HTML}{258F8F}
    \definecolor{ansi-white}{HTML}{C5C1B4}
    \definecolor{ansi-white-intense}{HTML}{A1A6B2}

    % commands and environments needed by pandoc snippets
    % extracted from the output of `pandoc -s`
    \providecommand{\tightlist}{%
      \setlength{\itemsep}{0pt}\setlength{\parskip}{0pt}}
    \DefineVerbatimEnvironment{Highlighting}{Verbatim}{commandchars=\\\{\}}
    % Add ',fontsize=\small' for more characters per line
    \newenvironment{Shaded}{}{}
    \newcommand{\KeywordTok}[1]{\textcolor[rgb]{0.00,0.44,0.13}{\textbf{{#1}}}}
    \newcommand{\DataTypeTok}[1]{\textcolor[rgb]{0.56,0.13,0.00}{{#1}}}
    \newcommand{\DecValTok}[1]{\textcolor[rgb]{0.25,0.63,0.44}{{#1}}}
    \newcommand{\BaseNTok}[1]{\textcolor[rgb]{0.25,0.63,0.44}{{#1}}}
    \newcommand{\FloatTok}[1]{\textcolor[rgb]{0.25,0.63,0.44}{{#1}}}
    \newcommand{\CharTok}[1]{\textcolor[rgb]{0.25,0.44,0.63}{{#1}}}
    \newcommand{\StringTok}[1]{\textcolor[rgb]{0.25,0.44,0.63}{{#1}}}
    \newcommand{\CommentTok}[1]{\textcolor[rgb]{0.38,0.63,0.69}{\textit{{#1}}}}
    \newcommand{\OtherTok}[1]{\textcolor[rgb]{0.00,0.44,0.13}{{#1}}}
    \newcommand{\AlertTok}[1]{\textcolor[rgb]{1.00,0.00,0.00}{\textbf{{#1}}}}
    \newcommand{\FunctionTok}[1]{\textcolor[rgb]{0.02,0.16,0.49}{{#1}}}
    \newcommand{\RegionMarkerTok}[1]{{#1}}
    \newcommand{\ErrorTok}[1]{\textcolor[rgb]{1.00,0.00,0.00}{\textbf{{#1}}}}
    \newcommand{\NormalTok}[1]{{#1}}
    
    % Additional commands for more recent versions of Pandoc
    \newcommand{\ConstantTok}[1]{\textcolor[rgb]{0.53,0.00,0.00}{{#1}}}
    \newcommand{\SpecialCharTok}[1]{\textcolor[rgb]{0.25,0.44,0.63}{{#1}}}
    \newcommand{\VerbatimStringTok}[1]{\textcolor[rgb]{0.25,0.44,0.63}{{#1}}}
    \newcommand{\SpecialStringTok}[1]{\textcolor[rgb]{0.73,0.40,0.53}{{#1}}}
    \newcommand{\ImportTok}[1]{{#1}}
    \newcommand{\DocumentationTok}[1]{\textcolor[rgb]{0.73,0.13,0.13}{\textit{{#1}}}}
    \newcommand{\AnnotationTok}[1]{\textcolor[rgb]{0.38,0.63,0.69}{\textbf{\textit{{#1}}}}}
    \newcommand{\CommentVarTok}[1]{\textcolor[rgb]{0.38,0.63,0.69}{\textbf{\textit{{#1}}}}}
    \newcommand{\VariableTok}[1]{\textcolor[rgb]{0.10,0.09,0.49}{{#1}}}
    \newcommand{\ControlFlowTok}[1]{\textcolor[rgb]{0.00,0.44,0.13}{\textbf{{#1}}}}
    \newcommand{\OperatorTok}[1]{\textcolor[rgb]{0.40,0.40,0.40}{{#1}}}
    \newcommand{\BuiltInTok}[1]{{#1}}
    \newcommand{\ExtensionTok}[1]{{#1}}
    \newcommand{\PreprocessorTok}[1]{\textcolor[rgb]{0.74,0.48,0.00}{{#1}}}
    \newcommand{\AttributeTok}[1]{\textcolor[rgb]{0.49,0.56,0.16}{{#1}}}
    \newcommand{\InformationTok}[1]{\textcolor[rgb]{0.38,0.63,0.69}{\textbf{\textit{{#1}}}}}
    \newcommand{\WarningTok}[1]{\textcolor[rgb]{0.38,0.63,0.69}{\textbf{\textit{{#1}}}}}
    
    
    % Define a nice break command that doesn't care if a line doesn't already
    % exist.
    \def\br{\hspace*{\fill} \\* }
    % Math Jax compatability definitions
    \def\gt{>}
    \def\lt{<}
    % Document parameters
    \title{chap1}
    
    
    

    % Pygments definitions
    
\makeatletter
\def\PY@reset{\let\PY@it=\relax \let\PY@bf=\relax%
    \let\PY@ul=\relax \let\PY@tc=\relax%
    \let\PY@bc=\relax \let\PY@ff=\relax}
\def\PY@tok#1{\csname PY@tok@#1\endcsname}
\def\PY@toks#1+{\ifx\relax#1\empty\else%
    \PY@tok{#1}\expandafter\PY@toks\fi}
\def\PY@do#1{\PY@bc{\PY@tc{\PY@ul{%
    \PY@it{\PY@bf{\PY@ff{#1}}}}}}}
\def\PY#1#2{\PY@reset\PY@toks#1+\relax+\PY@do{#2}}

\expandafter\def\csname PY@tok@gd\endcsname{\def\PY@tc##1{\textcolor[rgb]{0.63,0.00,0.00}{##1}}}
\expandafter\def\csname PY@tok@gu\endcsname{\let\PY@bf=\textbf\def\PY@tc##1{\textcolor[rgb]{0.50,0.00,0.50}{##1}}}
\expandafter\def\csname PY@tok@gt\endcsname{\def\PY@tc##1{\textcolor[rgb]{0.00,0.27,0.87}{##1}}}
\expandafter\def\csname PY@tok@gs\endcsname{\let\PY@bf=\textbf}
\expandafter\def\csname PY@tok@gr\endcsname{\def\PY@tc##1{\textcolor[rgb]{1.00,0.00,0.00}{##1}}}
\expandafter\def\csname PY@tok@cm\endcsname{\let\PY@it=\textit\def\PY@tc##1{\textcolor[rgb]{0.25,0.50,0.50}{##1}}}
\expandafter\def\csname PY@tok@vg\endcsname{\def\PY@tc##1{\textcolor[rgb]{0.10,0.09,0.49}{##1}}}
\expandafter\def\csname PY@tok@vi\endcsname{\def\PY@tc##1{\textcolor[rgb]{0.10,0.09,0.49}{##1}}}
\expandafter\def\csname PY@tok@vm\endcsname{\def\PY@tc##1{\textcolor[rgb]{0.10,0.09,0.49}{##1}}}
\expandafter\def\csname PY@tok@mh\endcsname{\def\PY@tc##1{\textcolor[rgb]{0.40,0.40,0.40}{##1}}}
\expandafter\def\csname PY@tok@cs\endcsname{\let\PY@it=\textit\def\PY@tc##1{\textcolor[rgb]{0.25,0.50,0.50}{##1}}}
\expandafter\def\csname PY@tok@ge\endcsname{\let\PY@it=\textit}
\expandafter\def\csname PY@tok@vc\endcsname{\def\PY@tc##1{\textcolor[rgb]{0.10,0.09,0.49}{##1}}}
\expandafter\def\csname PY@tok@il\endcsname{\def\PY@tc##1{\textcolor[rgb]{0.40,0.40,0.40}{##1}}}
\expandafter\def\csname PY@tok@go\endcsname{\def\PY@tc##1{\textcolor[rgb]{0.53,0.53,0.53}{##1}}}
\expandafter\def\csname PY@tok@cp\endcsname{\def\PY@tc##1{\textcolor[rgb]{0.74,0.48,0.00}{##1}}}
\expandafter\def\csname PY@tok@gi\endcsname{\def\PY@tc##1{\textcolor[rgb]{0.00,0.63,0.00}{##1}}}
\expandafter\def\csname PY@tok@gh\endcsname{\let\PY@bf=\textbf\def\PY@tc##1{\textcolor[rgb]{0.00,0.00,0.50}{##1}}}
\expandafter\def\csname PY@tok@ni\endcsname{\let\PY@bf=\textbf\def\PY@tc##1{\textcolor[rgb]{0.60,0.60,0.60}{##1}}}
\expandafter\def\csname PY@tok@nl\endcsname{\def\PY@tc##1{\textcolor[rgb]{0.63,0.63,0.00}{##1}}}
\expandafter\def\csname PY@tok@nn\endcsname{\let\PY@bf=\textbf\def\PY@tc##1{\textcolor[rgb]{0.00,0.00,1.00}{##1}}}
\expandafter\def\csname PY@tok@no\endcsname{\def\PY@tc##1{\textcolor[rgb]{0.53,0.00,0.00}{##1}}}
\expandafter\def\csname PY@tok@na\endcsname{\def\PY@tc##1{\textcolor[rgb]{0.49,0.56,0.16}{##1}}}
\expandafter\def\csname PY@tok@nb\endcsname{\def\PY@tc##1{\textcolor[rgb]{0.00,0.50,0.00}{##1}}}
\expandafter\def\csname PY@tok@nc\endcsname{\let\PY@bf=\textbf\def\PY@tc##1{\textcolor[rgb]{0.00,0.00,1.00}{##1}}}
\expandafter\def\csname PY@tok@nd\endcsname{\def\PY@tc##1{\textcolor[rgb]{0.67,0.13,1.00}{##1}}}
\expandafter\def\csname PY@tok@ne\endcsname{\let\PY@bf=\textbf\def\PY@tc##1{\textcolor[rgb]{0.82,0.25,0.23}{##1}}}
\expandafter\def\csname PY@tok@nf\endcsname{\def\PY@tc##1{\textcolor[rgb]{0.00,0.00,1.00}{##1}}}
\expandafter\def\csname PY@tok@si\endcsname{\let\PY@bf=\textbf\def\PY@tc##1{\textcolor[rgb]{0.73,0.40,0.53}{##1}}}
\expandafter\def\csname PY@tok@s2\endcsname{\def\PY@tc##1{\textcolor[rgb]{0.73,0.13,0.13}{##1}}}
\expandafter\def\csname PY@tok@nt\endcsname{\let\PY@bf=\textbf\def\PY@tc##1{\textcolor[rgb]{0.00,0.50,0.00}{##1}}}
\expandafter\def\csname PY@tok@nv\endcsname{\def\PY@tc##1{\textcolor[rgb]{0.10,0.09,0.49}{##1}}}
\expandafter\def\csname PY@tok@s1\endcsname{\def\PY@tc##1{\textcolor[rgb]{0.73,0.13,0.13}{##1}}}
\expandafter\def\csname PY@tok@dl\endcsname{\def\PY@tc##1{\textcolor[rgb]{0.73,0.13,0.13}{##1}}}
\expandafter\def\csname PY@tok@ch\endcsname{\let\PY@it=\textit\def\PY@tc##1{\textcolor[rgb]{0.25,0.50,0.50}{##1}}}
\expandafter\def\csname PY@tok@m\endcsname{\def\PY@tc##1{\textcolor[rgb]{0.40,0.40,0.40}{##1}}}
\expandafter\def\csname PY@tok@gp\endcsname{\let\PY@bf=\textbf\def\PY@tc##1{\textcolor[rgb]{0.00,0.00,0.50}{##1}}}
\expandafter\def\csname PY@tok@sh\endcsname{\def\PY@tc##1{\textcolor[rgb]{0.73,0.13,0.13}{##1}}}
\expandafter\def\csname PY@tok@ow\endcsname{\let\PY@bf=\textbf\def\PY@tc##1{\textcolor[rgb]{0.67,0.13,1.00}{##1}}}
\expandafter\def\csname PY@tok@sx\endcsname{\def\PY@tc##1{\textcolor[rgb]{0.00,0.50,0.00}{##1}}}
\expandafter\def\csname PY@tok@bp\endcsname{\def\PY@tc##1{\textcolor[rgb]{0.00,0.50,0.00}{##1}}}
\expandafter\def\csname PY@tok@c1\endcsname{\let\PY@it=\textit\def\PY@tc##1{\textcolor[rgb]{0.25,0.50,0.50}{##1}}}
\expandafter\def\csname PY@tok@fm\endcsname{\def\PY@tc##1{\textcolor[rgb]{0.00,0.00,1.00}{##1}}}
\expandafter\def\csname PY@tok@o\endcsname{\def\PY@tc##1{\textcolor[rgb]{0.40,0.40,0.40}{##1}}}
\expandafter\def\csname PY@tok@kc\endcsname{\let\PY@bf=\textbf\def\PY@tc##1{\textcolor[rgb]{0.00,0.50,0.00}{##1}}}
\expandafter\def\csname PY@tok@c\endcsname{\let\PY@it=\textit\def\PY@tc##1{\textcolor[rgb]{0.25,0.50,0.50}{##1}}}
\expandafter\def\csname PY@tok@mf\endcsname{\def\PY@tc##1{\textcolor[rgb]{0.40,0.40,0.40}{##1}}}
\expandafter\def\csname PY@tok@err\endcsname{\def\PY@bc##1{\setlength{\fboxsep}{0pt}\fcolorbox[rgb]{1.00,0.00,0.00}{1,1,1}{\strut ##1}}}
\expandafter\def\csname PY@tok@mb\endcsname{\def\PY@tc##1{\textcolor[rgb]{0.40,0.40,0.40}{##1}}}
\expandafter\def\csname PY@tok@ss\endcsname{\def\PY@tc##1{\textcolor[rgb]{0.10,0.09,0.49}{##1}}}
\expandafter\def\csname PY@tok@sr\endcsname{\def\PY@tc##1{\textcolor[rgb]{0.73,0.40,0.53}{##1}}}
\expandafter\def\csname PY@tok@mo\endcsname{\def\PY@tc##1{\textcolor[rgb]{0.40,0.40,0.40}{##1}}}
\expandafter\def\csname PY@tok@kd\endcsname{\let\PY@bf=\textbf\def\PY@tc##1{\textcolor[rgb]{0.00,0.50,0.00}{##1}}}
\expandafter\def\csname PY@tok@mi\endcsname{\def\PY@tc##1{\textcolor[rgb]{0.40,0.40,0.40}{##1}}}
\expandafter\def\csname PY@tok@kn\endcsname{\let\PY@bf=\textbf\def\PY@tc##1{\textcolor[rgb]{0.00,0.50,0.00}{##1}}}
\expandafter\def\csname PY@tok@cpf\endcsname{\let\PY@it=\textit\def\PY@tc##1{\textcolor[rgb]{0.25,0.50,0.50}{##1}}}
\expandafter\def\csname PY@tok@kr\endcsname{\let\PY@bf=\textbf\def\PY@tc##1{\textcolor[rgb]{0.00,0.50,0.00}{##1}}}
\expandafter\def\csname PY@tok@s\endcsname{\def\PY@tc##1{\textcolor[rgb]{0.73,0.13,0.13}{##1}}}
\expandafter\def\csname PY@tok@kp\endcsname{\def\PY@tc##1{\textcolor[rgb]{0.00,0.50,0.00}{##1}}}
\expandafter\def\csname PY@tok@w\endcsname{\def\PY@tc##1{\textcolor[rgb]{0.73,0.73,0.73}{##1}}}
\expandafter\def\csname PY@tok@kt\endcsname{\def\PY@tc##1{\textcolor[rgb]{0.69,0.00,0.25}{##1}}}
\expandafter\def\csname PY@tok@sc\endcsname{\def\PY@tc##1{\textcolor[rgb]{0.73,0.13,0.13}{##1}}}
\expandafter\def\csname PY@tok@sb\endcsname{\def\PY@tc##1{\textcolor[rgb]{0.73,0.13,0.13}{##1}}}
\expandafter\def\csname PY@tok@sa\endcsname{\def\PY@tc##1{\textcolor[rgb]{0.73,0.13,0.13}{##1}}}
\expandafter\def\csname PY@tok@k\endcsname{\let\PY@bf=\textbf\def\PY@tc##1{\textcolor[rgb]{0.00,0.50,0.00}{##1}}}
\expandafter\def\csname PY@tok@se\endcsname{\let\PY@bf=\textbf\def\PY@tc##1{\textcolor[rgb]{0.73,0.40,0.13}{##1}}}
\expandafter\def\csname PY@tok@sd\endcsname{\let\PY@it=\textit\def\PY@tc##1{\textcolor[rgb]{0.73,0.13,0.13}{##1}}}

\def\PYZbs{\char`\\}
\def\PYZus{\char`\_}
\def\PYZob{\char`\{}
\def\PYZcb{\char`\}}
\def\PYZca{\char`\^}
\def\PYZam{\char`\&}
\def\PYZlt{\char`\<}
\def\PYZgt{\char`\>}
\def\PYZsh{\char`\#}
\def\PYZpc{\char`\%}
\def\PYZdl{\char`\$}
\def\PYZhy{\char`\-}
\def\PYZsq{\char`\'}
\def\PYZdq{\char`\"}
\def\PYZti{\char`\~}
% for compatibility with earlier versions
\def\PYZat{@}
\def\PYZlb{[}
\def\PYZrb{]}
\makeatother


    % Exact colors from NB
    \definecolor{incolor}{rgb}{0.0, 0.0, 0.5}
    \definecolor{outcolor}{rgb}{0.545, 0.0, 0.0}



    
    % Prevent overflowing lines due to hard-to-break entities
    \sloppy 
    % Setup hyperref package
    \hypersetup{
      breaklinks=true,  % so long urls are correctly broken across lines
      colorlinks=true,
      urlcolor=urlcolor,
      linkcolor=linkcolor,
      citecolor=citecolor,
      }
    % Slightly bigger margins than the latex defaults
    
    \geometry{verbose,tmargin=1in,bmargin=1in,lmargin=1in,rmargin=1in}
    
    

    \begin{document}
    
    
    \maketitle
    
    

    
    \section{Chap1 介绍}\label{chap1-ux4ecbux7ecd}

任何机器学习问题可以描述为以下三个概念:

\begin{verbatim}
- 需要学习如何解决的任务为T。
- 为解决该任务所需要的经验为E。
- 度量解决任务执行效果的P。
\end{verbatim}

\subsection{线性分类}\label{ux7ebfux6027ux5206ux7c7b}

    \begin{Verbatim}[commandchars=\\\{\}]
{\color{incolor}In [{\color{incolor}23}]:} \PY{o}{\PYZpc{}}\PY{k}{matplotlib} inline 
         \PY{k+kn}{import} \PY{n+nn}{numpy} \PY{k+kn}{as} \PY{n+nn}{np}
         \PY{k+kn}{import} \PY{n+nn}{matplotlib.pyplot} \PY{k+kn}{as} \PY{n+nn}{plt}
         \PY{k+kn}{from} \PY{n+nn}{sklearn.cross\PYZus{}validation} \PY{k+kn}{import} \PY{n}{train\PYZus{}test\PYZus{}split}
         \PY{k+kn}{from} \PY{n+nn}{sklearn} \PY{k+kn}{import} \PY{n}{preprocessing}
         \PY{k+kn}{from} \PY{n+nn}{sklearn} \PY{k+kn}{import} \PY{n}{datasets}
         
         \PY{k+kn}{from} \PY{n+nn}{sklearn.linear\PYZus{}model} \PY{k+kn}{import} \PY{n}{SGDClassifier}
         
         \PY{n}{iris} \PY{o}{=} \PY{n}{datasets}\PY{o}{.}\PY{n}{load\PYZus{}iris}\PY{p}{(}\PY{p}{)}
         
         \PY{n}{X\PYZus{}iris}\PY{p}{,} \PY{n}{y\PYZus{}iris} \PY{o}{=} \PY{n}{iris}\PY{o}{.}\PY{n}{data}\PY{p}{,} \PY{n}{iris}\PY{o}{.}\PY{n}{target}
         
         \PY{c+c1}{\PYZsh{}这里只取前两个特征}
         \PY{n}{X}\PY{p}{,} \PY{n}{y} \PY{o}{=} \PY{n}{X\PYZus{}iris}\PY{p}{[}\PY{p}{:}\PY{p}{,} \PY{p}{:}\PY{l+m+mi}{2}\PY{p}{]}\PY{p}{,}\PY{n}{y\PYZus{}iris}
         
         \PY{c+c1}{\PYZsh{}分割训练和测试数据集}
         \PY{n}{X\PYZus{}train}\PY{p}{,}\PY{n}{X\PYZus{}test}\PY{p}{,}\PY{n}{y\PYZus{}train}\PY{p}{,}\PY{n}{y\PYZus{}test} \PY{o}{=} \PY{n}{train\PYZus{}test\PYZus{}split}\PY{p}{(}\PY{n}{X}\PY{p}{,}\PY{n}{y}\PY{p}{,}
                                                          \PY{n}{test\PYZus{}size}\PY{o}{=}\PY{l+m+mf}{0.25}\PY{p}{,}
                                                          \PY{n}{random\PYZus{}state}\PY{o}{=}\PY{l+m+mi}{33}\PY{p}{)}
         \PY{k}{print}\PY{p}{(}\PY{n}{X\PYZus{}train}\PY{o}{.}\PY{n}{shape}\PY{p}{,}\PY{n}{y\PYZus{}train}\PY{o}{.}\PY{n}{shape}\PY{p}{)}
         
         \PY{c+c1}{\PYZsh{}Standardize 数据归一化处理}
         \PY{n}{scaler} \PY{o}{=} \PY{n}{preprocessing}\PY{o}{.}\PY{n}{StandardScaler}\PY{p}{(}\PY{p}{)}\PY{o}{.}\PY{n}{fit}\PY{p}{(}\PY{n}{X\PYZus{}train}\PY{p}{)}
         \PY{n}{X\PYZus{}train} \PY{o}{=} \PY{n}{scaler}\PY{o}{.}\PY{n}{transform}\PY{p}{(}\PY{n}{X\PYZus{}train}\PY{p}{)}
         \PY{n}{X\PYZus{}test} \PY{o}{=} \PY{n}{scaler}\PY{o}{.}\PY{n}{transform}\PY{p}{(}\PY{n}{X\PYZus{}test}\PY{p}{)}
         
         \PY{c+c1}{\PYZsh{}绘制训练样本点图}
         \PY{n}{colors} \PY{o}{=} \PY{p}{[}\PY{l+s+s1}{\PYZsq{}}\PY{l+s+s1}{red}\PY{l+s+s1}{\PYZsq{}}\PY{p}{,}\PY{l+s+s1}{\PYZsq{}}\PY{l+s+s1}{greenyellow}\PY{l+s+s1}{\PYZsq{}}\PY{p}{,}\PY{l+s+s1}{\PYZsq{}}\PY{l+s+s1}{blue}\PY{l+s+s1}{\PYZsq{}}\PY{p}{]}
         \PY{k}{for} \PY{n}{i} \PY{o+ow}{in} \PY{n+nb}{xrange}\PY{p}{(}\PY{n+nb}{len}\PY{p}{(}\PY{n}{colors}\PY{p}{)}\PY{p}{)}\PY{p}{:}
             \PY{n}{xs} \PY{o}{=} \PY{n}{X\PYZus{}train}\PY{p}{[}\PY{p}{:}\PY{p}{,}\PY{l+m+mi}{0}\PY{p}{]}\PY{p}{[}\PY{n}{y\PYZus{}train} \PY{o}{==} \PY{n}{i}\PY{p}{]}
             \PY{n}{ys} \PY{o}{=} \PY{n}{X\PYZus{}train}\PY{p}{[}\PY{p}{:}\PY{p}{,}\PY{l+m+mi}{1}\PY{p}{]}\PY{p}{[}\PY{n}{y\PYZus{}train} \PY{o}{==} \PY{n}{i}\PY{p}{]}
             \PY{n}{plt}\PY{o}{.}\PY{n}{scatter}\PY{p}{(}\PY{n}{xs}\PY{p}{,}\PY{n}{ys}\PY{p}{,}\PY{n}{c}\PY{o}{=}\PY{n}{colors}\PY{p}{[}\PY{n}{i}\PY{p}{]}\PY{p}{)}
         
         \PY{n}{plt}\PY{o}{.}\PY{n}{legend}\PY{p}{(}\PY{n}{iris}\PY{o}{.}\PY{n}{target\PYZus{}names}\PY{p}{)}
         \PY{n}{plt}\PY{o}{.}\PY{n}{xlabel}\PY{p}{(}\PY{l+s+s1}{\PYZsq{}}\PY{l+s+s1}{Sepal length}\PY{l+s+s1}{\PYZsq{}}\PY{p}{)}
         \PY{n}{plt}\PY{o}{.}\PY{n}{ylabel}\PY{p}{(}\PY{l+s+s1}{\PYZsq{}}\PY{l+s+s1}{Sepal width}\PY{l+s+s1}{\PYZsq{}}\PY{p}{)}
         
         \PY{c+c1}{\PYZsh{}随机梯度下降分类器}
         \PY{n}{clf} \PY{o}{=} \PY{n}{SGDClassifier}\PY{p}{(}\PY{p}{)}
         \PY{n}{clf}\PY{o}{.}\PY{n}{fit}\PY{p}{(}\PY{n}{X\PYZus{}train}\PY{p}{,}\PY{n}{y\PYZus{}train}\PY{p}{)}
         
         \PY{c+c1}{\PYZsh{}系数,为3行,每类一个系数,每行两列,分别作为样本数据两列的系数}
         \PY{k}{print}\PY{p}{(}\PY{n}{clf}\PY{o}{.}\PY{n}{coef\PYZus{}}\PY{p}{)}
         \PY{c+c1}{\PYZsh{}y轴交点,也是3个,每类一个交点}
         \PY{k}{print}\PY{p}{(}\PY{n}{clf}\PY{o}{.}\PY{n}{intercept\PYZus{}}\PY{p}{)}
\end{Verbatim}


    \begin{Verbatim}[commandchars=\\\{\}]
((112, 2), (112,))
[[-37.50635483  16.42537437]
 [ -4.59180001   0.75101269]
 [  8.19869218  -1.35666809]]
[-18.63515639  -9.00188573  -4.08537094]

    \end{Verbatim}

    \begin{center}
    \adjustimage{max size={0.9\linewidth}{0.9\paperheight}}{output_1_1.png}
    \end{center}
    { \hspace*{\fill} \\}
    
    使用上面输出的数据可以在图上画出三条不同的直线,这三条直线用于分隔不同的分类。ceof\_的第一行为第一条直线的系数,a1,a2,intercept\_为第一条直线的b,b+a1\emph{x1+a2}x2=0,这里x1和x2为两个特征列,对应于图上的横轴和纵轴。

    \begin{Verbatim}[commandchars=\\\{\}]
{\color{incolor}In [{\color{incolor}24}]:} \PY{c+c1}{\PYZsh{}绘制这三条决策边界}
         
         \PY{c+c1}{\PYZsh{}图样x,y轴边界}
         \PY{n}{x\PYZus{}min}\PY{p}{,}\PY{n}{x\PYZus{}max} \PY{o}{=} \PY{n}{X\PYZus{}train}\PY{p}{[}\PY{p}{:}\PY{p}{,}\PY{l+m+mi}{0}\PY{p}{]}\PY{o}{.}\PY{n}{min}\PY{p}{(}\PY{p}{)} \PY{o}{\PYZhy{}} \PY{o}{.}\PY{l+m+mi}{5} \PY{p}{,}\PY{n}{X\PYZus{}train}\PY{p}{[}\PY{p}{:}\PY{p}{,}\PY{l+m+mi}{0}\PY{p}{]}\PY{o}{.}\PY{n}{max}\PY{p}{(}\PY{p}{)} \PY{o}{+} \PY{l+m+mf}{0.5}
         \PY{n}{y\PYZus{}min}\PY{p}{,}\PY{n}{y\PYZus{}max} \PY{o}{=} \PY{n}{X\PYZus{}train}\PY{p}{[}\PY{p}{:}\PY{p}{,}\PY{l+m+mi}{1}\PY{p}{]}\PY{o}{.}\PY{n}{min}\PY{p}{(}\PY{p}{)} \PY{o}{\PYZhy{}} \PY{o}{.}\PY{l+m+mi}{5} \PY{p}{,}\PY{n}{X\PYZus{}train}\PY{p}{[}\PY{p}{:}\PY{p}{,}\PY{l+m+mi}{1}\PY{p}{]}\PY{o}{.}\PY{n}{max}\PY{p}{(}\PY{p}{)} \PY{o}{+} \PY{l+m+mf}{0.5}
         
         \PY{n}{xs} \PY{o}{=} \PY{n}{np}\PY{o}{.}\PY{n}{arange}\PY{p}{(}\PY{n}{x\PYZus{}min}\PY{p}{,}\PY{n}{x\PYZus{}max}\PY{p}{,}\PY{l+m+mf}{0.5}\PY{p}{)}
         
         \PY{n}{fig}\PY{p}{,}\PY{n}{axes} \PY{o}{=} \PY{n}{plt}\PY{o}{.}\PY{n}{subplots}\PY{p}{(}\PY{l+m+mi}{1}\PY{p}{,}\PY{l+m+mi}{3}\PY{p}{)}
         \PY{n}{fig}\PY{o}{.}\PY{n}{set\PYZus{}size\PYZus{}inches}\PY{p}{(}\PY{l+m+mi}{12}\PY{p}{,}\PY{l+m+mi}{8}\PY{p}{)}
         \PY{k}{for} \PY{n}{i} \PY{o+ow}{in} \PY{p}{[}\PY{l+m+mi}{0}\PY{p}{,}\PY{l+m+mi}{1}\PY{p}{,}\PY{l+m+mi}{2}\PY{p}{]}\PY{p}{:}
             \PY{n}{axes}\PY{p}{[}\PY{n}{i}\PY{p}{]}\PY{o}{.}\PY{n}{set\PYZus{}aspect}\PY{p}{(}\PY{l+s+s1}{\PYZsq{}}\PY{l+s+s1}{equal}\PY{l+s+s1}{\PYZsq{}}\PY{p}{)}
             \PY{n}{axes}\PY{p}{[}\PY{n}{i}\PY{p}{]}\PY{o}{.}\PY{n}{set\PYZus{}title}\PY{p}{(}\PY{l+s+s1}{\PYZsq{}}\PY{l+s+s1}{Class }\PY{l+s+s1}{\PYZsq{}}\PY{o}{+}\PY{n+nb}{str}\PY{p}{(}\PY{n}{i}\PY{p}{)}\PY{o}{+}\PY{l+s+s1}{\PYZsq{}}\PY{l+s+s1}{ versus the rest}\PY{l+s+s1}{\PYZsq{}}\PY{p}{)}
             \PY{n}{axes}\PY{p}{[}\PY{n}{i}\PY{p}{]}\PY{o}{.}\PY{n}{set\PYZus{}xlabel}\PY{p}{(}\PY{l+s+s1}{\PYZsq{}}\PY{l+s+s1}{Sepal length}\PY{l+s+s1}{\PYZsq{}}\PY{p}{)}
             \PY{n}{axes}\PY{p}{[}\PY{n}{i}\PY{p}{]}\PY{o}{.}\PY{n}{set\PYZus{}ylabel}\PY{p}{(}\PY{l+s+s1}{\PYZsq{}}\PY{l+s+s1}{Sepal width}\PY{l+s+s1}{\PYZsq{}}\PY{p}{)}
             \PY{n}{axes}\PY{p}{[}\PY{n}{i}\PY{p}{]}\PY{o}{.}\PY{n}{set\PYZus{}xlim}\PY{p}{(}\PY{n}{x\PYZus{}min}\PY{p}{,} \PY{n}{x\PYZus{}max}\PY{p}{)}
             \PY{n}{axes}\PY{p}{[}\PY{n}{i}\PY{p}{]}\PY{o}{.}\PY{n}{set\PYZus{}ylim}\PY{p}{(}\PY{n}{y\PYZus{}min}\PY{p}{,}\PY{n}{y\PYZus{}max}\PY{p}{)}
             \PY{n}{fig}\PY{o}{.}\PY{n}{sca}\PY{p}{(}\PY{n}{axes}\PY{p}{[}\PY{n}{i}\PY{p}{]}\PY{p}{)}
             \PY{n}{plt}\PY{o}{.}\PY{n}{scatter}\PY{p}{(}\PY{n}{X\PYZus{}train}\PY{p}{[}\PY{p}{:}\PY{p}{,}\PY{l+m+mi}{0}\PY{p}{]}\PY{p}{,}\PY{n}{X\PYZus{}train}\PY{p}{[}\PY{p}{:}\PY{p}{,}\PY{l+m+mi}{1}\PY{p}{]}\PY{p}{,}\PY{n}{c}\PY{o}{=}\PY{n}{y\PYZus{}train}\PY{p}{,}
                        \PY{n}{cmap}\PY{o}{=}\PY{n}{plt}\PY{o}{.}\PY{n}{cm}\PY{o}{.}\PY{n}{prism}\PY{p}{)}
             \PY{n}{ys} \PY{o}{=} \PY{p}{(}\PY{o}{\PYZhy{}}\PY{n}{clf}\PY{o}{.}\PY{n}{intercept\PYZus{}}\PY{p}{[}\PY{n}{i}\PY{p}{]} \PY{o}{\PYZhy{}} \PY{n}{xs} \PY{o}{*} \PY{n}{clf}\PY{o}{.}\PY{n}{coef\PYZus{}}\PY{p}{[}\PY{n}{i}\PY{p}{,}\PY{l+m+mi}{0}\PY{p}{]}\PY{p}{)}\PY{o}{/}\PY{n}{clf}\PY{o}{.}\PY{n}{coef\PYZus{}}\PY{p}{[}\PY{n}{i}\PY{p}{,}\PY{l+m+mi}{1}\PY{p}{]}
             \PY{n}{plt}\PY{o}{.}\PY{n}{plot}\PY{p}{(}\PY{n}{xs}\PY{p}{,}\PY{n}{ys}\PY{p}{,}\PY{l+s+s1}{\PYZsq{}}\PY{l+s+s1}{b}\PY{l+s+s1}{\PYZsq{}}\PY{p}{)}
\end{Verbatim}


    \begin{center}
    \adjustimage{max size={0.9\linewidth}{0.9\paperheight}}{output_3_0.png}
    \end{center}
    { \hspace*{\fill} \\}
    
    模型训练完毕。但是查看曲线只有第一类的分类效果比较好。现在使用数据进行测试:

    \begin{Verbatim}[commandchars=\\\{\}]
{\color{incolor}In [{\color{incolor}25}]:} \PY{k}{print}\PY{p}{(}\PY{n}{clf}\PY{o}{.}\PY{n}{predict}\PY{p}{(}\PY{n}{scaler}\PY{o}{.}\PY{n}{transform}\PY{p}{(}\PY{p}{[}\PY{p}{[}\PY{l+m+mf}{4.7}\PY{p}{,}\PY{l+m+mf}{3.1}\PY{p}{]}\PY{p}{]}\PY{p}{)}\PY{p}{)}\PY{p}{)}
\end{Verbatim}


    \begin{Verbatim}[commandchars=\\\{\}]
[0]

    \end{Verbatim}

    \texttt{predict}返回预测的分类结果列表(这里只有一个样本)。\texttt{predict}方法只返回三分类中的分类结果。在这里选择的是距离最远的边界。可以通过\texttt{decision\_function}来检查:

    \begin{Verbatim}[commandchars=\\\{\}]
{\color{incolor}In [{\color{incolor}26}]:} \PY{k}{print}\PY{p}{(}\PY{n}{clf}\PY{o}{.}\PY{n}{decision\PYZus{}function}\PY{p}{(}\PY{n}{scaler}\PY{o}{.}\PY{n}{transform}\PY{p}{(}\PY{p}{[}\PY{p}{[}\PY{l+m+mf}{4.7}\PY{p}{,}\PY{l+m+mf}{3.1}\PY{p}{]}\PY{p}{]}\PY{p}{)}\PY{p}{)}\PY{p}{)}
\end{Verbatim}


    \begin{Verbatim}[commandchars=\\\{\}]
[[ 30.15506368  -3.14759504 -14.5397346 ]]

    \end{Verbatim}

    \subsection{评估分类结果}\label{ux8bc4ux4f30ux5206ux7c7bux7ed3ux679c}

\texttt{精度(accuracy)}:给定数据集给分类器,计算准确分类结果的占比。

    \begin{Verbatim}[commandchars=\\\{\}]
{\color{incolor}In [{\color{incolor}27}]:} \PY{k+kn}{from} \PY{n+nn}{sklearn} \PY{k+kn}{import} \PY{n}{metrics}
         \PY{n}{y\PYZus{}train\PYZus{}pred} \PY{o}{=} \PY{n}{clf}\PY{o}{.}\PY{n}{predict}\PY{p}{(}\PY{n}{X\PYZus{}train}\PY{p}{)}
         \PY{n}{metrics}\PY{o}{.}\PY{n}{accuracy\PYZus{}score}\PY{p}{(}\PY{n}{y\PYZus{}train}\PY{p}{,}\PY{n}{y\PYZus{}train\PYZus{}pred}\PY{p}{)}
\end{Verbatim}


\begin{Verbatim}[commandchars=\\\{\}]
{\color{outcolor}Out[{\color{outcolor}27}]:} 0.6696428571428571
\end{Verbatim}
            
    像上面这样在训练集上进行精度测量是不合适的。因为是基于这些数据构建的模型,有可能对于这些数据模型己经被训练得非常适合了,但对于其它数据则效果不好。这种现象被称为\texttt{过拟合(overfitting)}。如果在训练集上进行检测,你将检测不到过拟合。因此,\texttt{不要在训练集上进行检测}。

应该使用之前保留的测试数据集进行检测:

    \begin{Verbatim}[commandchars=\\\{\}]
{\color{incolor}In [{\color{incolor}28}]:} \PY{n}{y\PYZus{}pred} \PY{o}{=} \PY{n}{clf}\PY{o}{.}\PY{n}{predict}\PY{p}{(}\PY{n}{X\PYZus{}test}\PY{p}{)}
         \PY{n}{metrics}\PY{o}{.}\PY{n}{accuracy\PYZus{}score}\PY{p}{(}\PY{n}{y\PYZus{}test}\PY{p}{,}\PY{n}{y\PYZus{}pred}\PY{p}{)}
\end{Verbatim}


\begin{Verbatim}[commandchars=\\\{\}]
{\color{outcolor}Out[{\color{outcolor}28}]:} 0.71052631578947367
\end{Verbatim}
            
    通常在测试集上的精度会低于训练集上的精度,因为模型是在训练集上训练的。我们的目标是在训练集上避免过拟合,以保证模型的\texttt{泛化(generalization)}能力。

精度检测在各分类样本数量差别不大,即样本是\texttt{均匀分布(uniform\ distribution)}时是一个较好的度量方法。但是如果样本是\texttt{偏态分布(skewed\ distribution)}时,分类器在对样本中的主分类数据进行检测时将得到一个极高精度,这是一种非常幼稚的方法。

sklearn中包含了多种评估函数;三个最流行的是:\texttt{precision}、\texttt{recall},\texttt{F1-score(f-measure)}。将预测值用正负例\texttt{P}、\texttt{N}表示,实际值用\texttt{T}、\texttt{F}表示,它们的计算公式为:
- \texttt{accuracy\ =\ (TP\ +\ TN)\ /\ m} -
\texttt{precision\ =\ TP\ /\ (TP\ +\ FP)} -
\texttt{recall\ =\ TP\ /\ (TP\ +\ FN)} -
\texttt{f1score\ =\ 2\ *\ precision\ *\ recall/(precision\ +\ recall)}

    \begin{Verbatim}[commandchars=\\\{\}]
{\color{incolor}In [{\color{incolor}29}]:} \PY{k}{print}\PY{p}{(}\PY{n}{metrics}\PY{o}{.}\PY{n}{classification\PYZus{}report}\PY{p}{(}\PY{n}{y\PYZus{}test}\PY{p}{,}\PY{n}{y\PYZus{}pred}\PY{p}{,}\PY{n}{target\PYZus{}names}\PY{o}{=}\PY{n}{iris}\PY{o}{.}\PY{n}{target\PYZus{}names}\PY{p}{)}\PY{p}{)}
\end{Verbatim}


    \begin{Verbatim}[commandchars=\\\{\}]
             precision    recall  f1-score   support

     setosa       1.00      1.00      1.00         8
 versicolor       0.00      0.00      0.00        11
  virginica       0.63      1.00      0.78        19

avg / total       0.53      0.71      0.60        38


    \end{Verbatim}

    从上面的结果可以看出:

\begin{itemize}
\item
  setosa这一分类的precision和recall都是1.0。precision
  1.0表示所有被分类器分类为setosa的样例都是真正属于setosa分类的。recall
  1.0表示所有setosa的样例都被标记为了setosa(全都被找了出来)。
\item
  对于versicolor这一分类,结果则不太好:precision只有43\%,即被分类器标识为versicolor的样例中只有43\%是真正属于versicolor类的。recall为27\%,在所有versicolor样本中只有27\%的样本被正确的分类了。
\end{itemize}

另一个分类结果评估工具是混淆矩阵(特别是于多分类问题),好的分类器的混淆矩阵应该是一个对角矩阵。上面的分类器的混淆矩阵如下:

    \begin{Verbatim}[commandchars=\\\{\}]
{\color{incolor}In [{\color{incolor}30}]:} \PY{k}{print} \PY{n}{metrics}\PY{o}{.}\PY{n}{confusion\PYZus{}matrix}\PY{p}{(}\PY{n}{y\PYZus{}test}\PY{p}{,}\PY{n}{y\PYZus{}pred}\PY{p}{)}
\end{Verbatim}


    \begin{Verbatim}[commandchars=\\\{\}]
[[ 8  0  0]
 [ 0  0 11]
 [ 0  0 19]]

    \end{Verbatim}

    上面模型的评估效果在setosa上比较准确,而对于另外两类则不完全准确。一个原因是因为我们对训练集和测试集的划分,划分的方式(通常是随机的)会影响测试的结果。

交叉验证法可以降低由于测试/训练集划分带来的的影响,得到更好的效果。k拆交叉验证法的步骤如下:

\begin{enumerate}
\def\labelenumi{\arabic{enumi}.}
\tightlist
\item
  将数据集划分为k个不同的子集
\item
  在k-1个子集上创建k个不同的模型,并使用剩下的1个子集上进行测试
\item
  测试在k个模型上的性能,取平均值
\end{enumerate}

    \begin{Verbatim}[commandchars=\\\{\}]
{\color{incolor}In [{\color{incolor}37}]:} \PY{k+kn}{from} \PY{n+nn}{sklearn.cross\PYZus{}validation} \PY{k+kn}{import} \PY{n}{cross\PYZus{}val\PYZus{}score}\PY{p}{,} \PY{n}{KFold}
         \PY{k+kn}{from} \PY{n+nn}{sklearn.pipeline} \PY{k+kn}{import} \PY{n}{Pipeline}
         \PY{k+kn}{from} \PY{n+nn}{sklearn.preprocessing} \PY{k+kn}{import} \PY{n}{StandardScaler}
         
         \PY{c+c1}{\PYZsh{}创建标准化和线性模型的pipeline}
         \PY{n}{clf} \PY{o}{=} \PY{n}{Pipeline}\PY{p}{(}\PY{p}{[}
             \PY{p}{(}\PY{l+s+s1}{\PYZsq{}}\PY{l+s+s1}{scaler}\PY{l+s+s1}{\PYZsq{}}\PY{p}{,}\PY{n}{StandardScaler}\PY{p}{(}\PY{p}{)}\PY{p}{)}\PY{p}{,}
             \PY{p}{(}\PY{l+s+s1}{\PYZsq{}}\PY{l+s+s1}{linear\PYZus{}model}\PY{l+s+s1}{\PYZsq{}}\PY{p}{,}\PY{n}{SGDClassifier}\PY{p}{(}\PY{p}{)}\PY{p}{)}\PY{p}{]}\PY{p}{)}
         
         \PY{c+c1}{\PYZsh{}k拆交叉验证}
         \PY{n}{cv} \PY{o}{=} \PY{n}{KFold}\PY{p}{(}\PY{n}{X}\PY{o}{.}\PY{n}{shape}\PY{p}{[}\PY{l+m+mi}{0}\PY{p}{]}\PY{p}{,} \PY{l+m+mi}{5}\PY{p}{,} \PY{n}{shuffle}\PY{o}{=}\PY{n+nb+bp}{True}\PY{p}{,} \PY{n}{random\PYZus{}state}\PY{o}{=}\PY{l+m+mi}{33}\PY{p}{)}
         
         \PY{n}{scores} \PY{o}{=} \PY{n}{cross\PYZus{}val\PYZus{}score}\PY{p}{(}\PY{n}{clf}\PY{p}{,}\PY{n}{X}\PY{p}{,}\PY{n}{y}\PY{p}{,}\PY{n}{cv}\PY{o}{=}\PY{n}{cv}\PY{p}{)}
         \PY{k}{print} \PY{n}{scores}
         
         \PY{c+c1}{\PYZsh{}scores = cross\PYZus{}val\PYZus{}score(clf,X,y,cv=6)}
         \PY{c+c1}{\PYZsh{}print scores}
         
         \PY{k+kn}{from} \PY{n+nn}{scipy.stats} \PY{k+kn}{import} \PY{n}{sem}
         \PY{k}{def} \PY{n+nf}{mean\PYZus{}score}\PY{p}{(}\PY{n}{scores}\PY{p}{)}\PY{p}{:}
             \PY{k}{return} \PY{p}{(}\PY{l+s+s2}{\PYZdq{}}\PY{l+s+s2}{Mean score: \PYZob{}0:.3f\PYZcb{} (+/\PYZhy{} \PYZob{}1:.3f\PYZcb{})}\PY{l+s+s2}{\PYZdq{}}\PY{p}{)}\PY{o}{.}\PY{n}{format}\PY{p}{(}\PY{n}{np}\PY{o}{.}\PY{n}{mean}\PY{p}{(}\PY{n}{scores}\PY{p}{)}\PY{p}{,}\PY{n}{sem}\PY{p}{(}\PY{n}{scores}\PY{p}{)}\PY{p}{)}
         
         \PY{k}{print} \PY{n}{mean\PYZus{}score}\PY{p}{(}\PY{n}{scores}\PY{p}{)}
         
         \PY{c+c1}{\PYZsh{}模型平均精度为0.747}
\end{Verbatim}


    \begin{Verbatim}[commandchars=\\\{\}]
[ 0.7         0.7         0.8         0.83333333  0.7       ]
Mean score: 0.747 (+/- 0.029)

    \end{Verbatim}

    \subsection{机器学习分类}\label{ux673aux5668ux5b66ux4e60ux5206ux7c7b}

\begin{itemize}
\item
  监督学习:使用学习算法从训练数据训练模型,并使用该模型计算目标数据的分类。
\item
  拟合
\item
  非监督学习:聚类
\end{itemize}

\subsection{机器学习相关的重要概念}\label{ux673aux5668ux5b66ux4e60ux76f8ux5173ux7684ux91cdux8981ux6982ux5ff5}


    % Add a bibliography block to the postdoc
    
    
    
    \end{document}
